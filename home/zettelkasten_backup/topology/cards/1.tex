%%% Propiedades de la imagen e imagen inversa
%% tags: propiedades, imagen, imagen inversa

\zheader{Propiedades de conjuntos y funciones}{1}{propiedades, imagen, imagen inversa}

\begin{theorem}
Sean $f : X \longrightarrow Y$ una función, $A, A_1, A_2, \{ A_{\alpha} \}_{\alpha \in I}$ subconjuntos de $X$ y $B, B_1, B_2, \{ B_{\beta} \}_{\beta \in J}$ subconjuntos de $Y$, se tiene que

\begin{multicols}{2}
    \begin{enumerate}[label=\textnormal{(\roman*)}, align=left, labelwidth=2.0em, leftmargin=3.0em]
    \item $f(X - A) \subset Y - f(A)$ si $f$ es inyectiva,
    \item $Y - f(A) \subset f(X - A)$ si $f$ es suprayectiva,
    \item $f^{-1}(Y - B) = X - f^{-1}(B)$,
    \item $f(f^{-1}(B)) \subset B$,
    \item $B \subset f(f^{-1}(B))$ si $f$ es suprayectiva,
    \item $A \subset f^{-1}(f(A))$,
    \item $f^{-1}(f(A)) \subset A$ si $f$ es inyectiva,
    \columnbreak
    \item $f \left( \bigcup_{\alpha \in I} A_{\alpha} \right) = \bigcup_{\alpha \in I} f(A_{\alpha})$,
    \item $f \left( \bigcap_{\alpha \in I} A_{\alpha} \right) \subset \bigcap_{\alpha \in I} f(A_{\alpha})$,
    \item $\bigcap_{\alpha \in I} f(A_{\alpha}) \subset f \left( \bigcap_{\alpha \in I} A_{\alpha} \right)$ si $f$ es inyectiva,
    \item $f^{-1} \left( \bigcup_{\beta \in J} B_{\alpha} \right) = \bigcup_{\alpha \in J} f^{-1}(B_{\beta})$,
    \item $f^{-1}\left( \bigcap_{\beta \in J} B_{\beta} \right) = \bigcap_{\alpha \in J} f^{-1}(B_{\beta})$,
    \item $A_1 \subset A_2$ implica $f(A_1) \subset f(A_2)$,
    \item $B_1 \subset B_2$ implica $f^{-1}(B_1) \subset f^{-1}(B_2)$.
    \end{enumerate}
\end{multicols}
\end{theorem}

\begin{proof}
Pendiente.
\end{proof}

\begin{equation*}
    \sum_{i=1}^{\infty} \frac{1}{i^2} = \frac{\pi^2}{6}.
\end{equation*}
