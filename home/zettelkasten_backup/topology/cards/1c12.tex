%%% Caracterización de identificaciones
%% tags: caracterización, compatibilidad, identificación

\zheader{Caracterización de identificaciones}{1c12}{caracterización, compatibilidad, identificación}

\begin{definition}
Dada una función $p : X \longrightarrow \overline{X}$, se dice que otra función $f : X \longrightarrow Y$ es compatible con $p$ si $p(x) = p(x')$ implica que $f(x) = f(x')$, para cada $x, x' \in X$.
\end{definition}

\begin{theorem}
Sea $p : X \longrightarrow \overline{X}$ continua y suprayectiva. Entonces $p$ es identificación si y sólo si para cada función continua $f : X \longrightarrow Y$ compatible con $p$, existe una única función continua $\overline{f} : \overline{X} \longrightarrow Y$ tal que $\overline{f} \circ p = f$.
\bigskip

\adjustbox{scale=1.2,center}{
    \begin{tikzcd}[row sep=large,column sep=large]
        X \arrow[swap]{d}{p} \arrow{r}{f} & Y \\
        \overline{X} \arrow[swap,dashed]{ru}{\overline{f}}
    \end{tikzcd}
}
\end{theorem}

\begin{definition}
En la definición anterior, se dice que $\overline{f}$ es el resultado de pasar $f$ al cociente.
\end{definition}
