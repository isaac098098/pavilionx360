%%% Definición de topología de identificación
%% tags: generación, definición

\zheader{Definición de topología de identificación}{1b}{generación, definición}

\begin{definition}
Dados un espacio topológico $X$, un conjunto $Y$ y una función $f : X \longrightarrow Y$, se puede dotar a $Y$ con una topología, a saber, $\{ A \subset Y \mid f^{-1}(Y) \text{ es abierto en } X \}$. A esta topología se le llamará \emph{\bfseries topología de identificación} o \emph{\bfseries topología coinducida} en $Y$ por $X$ a través de $f$.
\end{definition}

\begin{definition}
Si $X$ y $Y$ son espacios topológicos y $f : X \longrightarrow Y$ es una función, se dice que $f$ es una \emph{\bfseries identificación} si la topología de $Y$ es la topología coinducida por $f$.
\end{definition}
