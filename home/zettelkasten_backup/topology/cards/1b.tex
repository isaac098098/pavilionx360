%%% Definición de relación de equivalencia
%% tags: definición, relación binaria, relación de equivalencia

\zheader{Definición de relación de equivalencia}{1b}{definición, relación binaria, relación de equivalencia}

\begin{definition}
Una \textbf{relación binaria} $R$ en un conjunto $X$ es cualquier subconjunto $R \subset X \times X$. Si $(x,y) \in R$, se escribirá $x \, R \, y$.
\end{definition}

\begin{definition}
Una relación binaria $R$ en un conjunto $X$ se dice \textbf{relación de equivalencia} si
\begin{enumerate}[label=\textnormal{(\roman*)}]
\item $\forall x \in X$, $x \, R \, x$,
\item $x \, R \, y \implies y \, R \, x$,
\item $x \, R \, y \land y \, R \, z \implies x \, R \, z$.
\end{enumerate}
Si $x \, R \, y$ se dice que $x$ y $y$ son \textbf{equivalentes}. Las relaciones de equivalencia se usan generalmente para considerar a todos los elementos de un conjunto con alguna propiedad como una sola entidad.
\end{definition}
