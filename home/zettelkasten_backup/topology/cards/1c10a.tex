%%% Propiedades de saturación
%% tags: definición, saturación, identificación

\zheader{Propiedades de saturación}{1c10a}{definición, saturación, identificación}

\begin{definition}
Si $p : X \longrightarrow X/\sim$ es una aplicación cociente y $A \subset X$, se define la \textbf{saturación} de $A$ como el conjunto $p^{-1}(p(A))$, que contiene a todos los puntos de $A$ y a todos los puntos en $X$ equivalentes a algún punto de $A$. Se dice que $A$ es \textbf{saturado} si $A = p^{-1}(p(A))$.
\end{definition}

\begin{proposition}
Sea $A \subset X$ un conjunto saturado respecto a una relaión de equivalencia $\sim$ y sea $p$ la respectiva aplicación cociente. Se tiene que
\begin{enumerate}[label=\textnormal{(\roman*)}]
\item Si $A \subset X$ es abierto o cerrado, entonces $p|_A : A \longrightarrow p(A)$ es una identificación.
\item Si $p$ es abierta o cerrada, entonces $p|_A : A \longrightarrow p(A)$ es una identificación.
\end{enumerate}
\end{proposition}

\begin{proof}
({\scshape\romannumeral 1}) Como $A$ es saturado, entonces $A = p^{-1}(p(A))$ y dado que $p$ es identificación y $A$ es abierto, $p(A)$ debe ser abierto en $X/\sim$. Y nuevamente, como $A = p^{-1}(p(A))$, entonces $p|_A : A \longrightarrow p(A)$ es una identificación por \hyperref[card:1c8]{\textsf{1c8}}.
\bigskip

({\scshape\romannumeral 2})
\end{proof}
