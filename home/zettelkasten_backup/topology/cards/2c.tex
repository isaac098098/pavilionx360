%%% Definición de suma topológica
%% tags: definición, suma topológica, unión disjunta, generación

\zheader{Definición de suma topológica}{2c}{definición, suma topológica, unión disjunta}

\begin{definition}
Dada una familia de espacios topológicos $\{ X_{\lambda} \}_{\lambda \in \Lambda}$, se puede generar un nuevo espacio topológico a partir de su unión ajena $X = \coprod_{\lambda \in \Lambda} X_{\lambda}$ definida en \hyperref[card:1a]{\textsf{1a}}. Considerénse las inclusiones $i_{\mu} : X_{\mu} \longrightarrow X, \mu \in \Lambda$ y sea $\mathcal{T}_{\mu}$ la topología coinducida en $X$ por $X_{\mu}$ a través de $i_{\mu}$. Se tiene que $\mathcal{S} = \bigcap_{\lambda \in \Lambda} \mathcal{T}_{\lambda}$ también es una topología sobre $X$. Esta topología se llamará \textbf{topología de la suma} en $X$. Al espacio $X$ con esta topología se le llamará \textbf{suma topológica} de los espacios $X_{\lambda}$.
\end{definition}
