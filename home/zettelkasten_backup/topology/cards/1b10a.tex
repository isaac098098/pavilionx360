%%% Propiedades de saturación
%% tags: definición, saturación, identificación

\zheader{Propiedades de saturación}{1b10a}{definición, saturación, identificación}

\begin{definition}
Si $p : X \longrightarrow X/\sim$ es una aplicación cociente y $A \subset X$, se define la \textbf{saturación} de $A$ como el conjunto $p^{-1}(p(A))$, que contiene a todos los puntos de $A$ y a todos los puntos en $X$ equivalentes a algún punto de $A$. Se dice que $A$ es \textbf{saturado} si $A = p^{-1}(p(A))$.
\end{definition}

\begin{proposition}
Si $A \subset X$ es abierto o cerrado y saturado respecto a una relación $\sim$ y $p$ es la respectiva aplicación cociente, entonces $p|_A : A \longrightarrow p(A)$ es una identificación.
\end{proposition}

\begin{proof}
Corolario de \hyperref[card:1b8]{\textsf{1b8}}.
\end{proof}

\begin{proposition}
Si $A \subset X$ es saturado respecto a una relación $\sim$, $p : X \longrightarrow X/\sim$ es la respectiva aplicación cociente y $p$ es una aplicación abierta o cerrada, entonces $p|_A : A \longrightarrow p(A)$ es una identificación.
\end{proposition}

\begin{proof}
Pendiente.
\end{proof}
