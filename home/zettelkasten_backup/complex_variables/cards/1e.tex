%%% Proyección estereográfica
%% tags: proyección estereográfica, esfera de Riemann

\zheader{Proyección estereográfica}{1e}{proyección estereográfica, esfera de Riemann}

\begin{definition}
Sea $S = \{ (x_1,x_2,x_3) \in \mathbb{R}^3 \mid x_1^2+x_2^2+x_3^2 = 1 \}$ y sea $N = (0,0,1)$. La función
\begin{align*}
    Z : \mathbb{C} & \longrightarrow S - \{ N \} \\
    z & \longmapsto \left( \frac{2 \text{Re}(z)}{|z|^2+1}, \frac{2 \text{Im}(z)}{|z|^2+1}, \frac{|z|^2-1}{|z|^2+1} \right)
\end{align*}
es una biyección y se le llamará \textbf{proyección esteteográfica}. A $S$ en este contexto se le suele llamar \textbf{esfera de Riemann}. La inversa de $Z$ está dada por
\begin{align*}
    z : S - \{ N \} & \longrightarrow \mathbb{C} \\
    (x_1,x_2,x_3) & \longmapsto \frac{x_1 + i x_2}{1 - x_3}.
\end{align*}
\end{definition}
