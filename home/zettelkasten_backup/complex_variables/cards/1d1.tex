%%% Determinación del argumento principal
%% tags: argumento principal

\zheader{Determinación del argumento principal}{1d1}{argumento principal}
\bigskip

Se tiene, de la definición anterior, que
\begin{equation*}
     \sin \Phi = \frac{y}{\sqrt{x^2 + y^2}},\quad \cos \Phi = \frac{x}{\sqrt{x^2 + y^2}}\quad \text{ y } \quad\tan \Phi = \frac{y}{x}.
\end{equation*}

Para obtener el argumento principal de $z$ a partir de la tangente, se debe considerar la convención acordada. Si existe el valor $y/x$, se considerará siempre $-\pi/2 < \arctan(y/x) < \pi/2$. Sea $\omega = \arctan(y/x)$.
\begin{multicols}{2}
\begin{enumerate}[label=\textnormal{(\roman*)}]
\item Para la primera convención, se tiene 
    \begin{equation*}
        \arg(z) = \begin{cases}
            \hfil \omega & \text{si } x>0, \\
            \hfil \omega + \pi & \text{si } x<0, y \ge 0, \\
            \hfil \omega - \pi & \text{si } x<0, y<0, \\
            \hfil \pi/2 & \text{si } x=0, y>0, \\
            \hfil -\pi/2 & \text{si } x=0, y<0.
        \end{cases}
    \end{equation*}
\item Para la segunda, se tiene 
    \begin{equation*}
        \arg(z) = \begin{cases}
            \hfil \omega + \pi & \text{si } x>0, \\
            \hfil \omega + 2 \pi & \text{si } x<0, y \ge 0, \\
            \hfil \omega + \pi & \text{si } x<0, y<0, \\
            \hfil \pi/2 & \text{si } x=0, y>0, \\
            \hfil 3 \pi/2 & \text{si } x=0, y<0.
        \end{cases}
    \end{equation*}
\end{enumerate}
\end{multicols}
