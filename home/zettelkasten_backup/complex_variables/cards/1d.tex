%%% Forma polar de un complejo
%% tags: definición, argumento, forma polar

\zheader{Forma polar de un complejo}{1d}{definición, argumento, forma polar}

\begin{definition}
Sea $z = x + i y \in \mathbb{C}$ y considere el vector $\vv{OP}$ que une al punto $O = (0,0)$ con el punto $P = (x,y)$ en el plano cartesiano. La longitud del vector $\vv{OP}$ es igual a $|z|$ y se suele denotar también como $r$. Un ángulo entre el eje horizontal positivo y el vector $\vv{OP}$, considerado positivo en el sentido de las manecillas del reloj y negativo en otro caso, se llamará un \textbf{argumento} de $z$ y está definido solo para $z \ne 0$. Es claro que no existe un sólo único argumento de $|z|$, pues si $\Phi$ es un argumento de $|z|$, también lo son cualesquiera de los números $\Phi + 2 \pi n, n \in \mathbb{Z}$. Se suelen usar dos convenciones para determinar un único argumento, como
\begin{enumerate}[label=\textnormal{(\roman*)}]
\item aquel ángulo $\varphi$ tal que $-\pi < \varphi \le \pi$,
\item o aquel ángulo $\varphi$ tal que $0 \le \varphi < 2 \pi$.
\end{enumerate}

En cualquiera de estas convenciones, a $\varphi$ se le llamará \textbf{argumento principal} de $z$, también denotado $\arg(z)$. Al par $(r, \Phi)$, donde $\Phi$ es cualquier argumento de $z$, se les llamará \textbf{coordenadas polares} del número complejo $z$.
\end{definition}
