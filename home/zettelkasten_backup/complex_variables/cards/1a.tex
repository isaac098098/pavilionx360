%%% Forma binómica de un complejo
%% tags: definición, forma binómica

\zheader{Forma binómica de un complejo}{1a}{definición, forma binómica}

\begin{definition}
Obsérvese que si $(a,b) \in \mathbb{C}$, entonces $(a,b) = (a,0) + (0,1)(b,0) = (a,0) + i (b,0)$, donde $i = (0,1)$. De ahora en adelante, se hará la convención de escribir $x$ en lugar de $(x,0)$, $x \in \mathbb{R}$, de tal manera que $(a,b) = a + i b$. Esta forma de escribir números complejos se llamará \textbf{forma binómica}. Se escribirá $i a$ en vez de $0 + i a$, $a$ en vez de $a + i 0$ y $a \pm i$ en vez de $a \pm i 1$.
\end{definition}
